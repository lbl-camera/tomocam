% irt.tex
% documentation for image reconstruction toolbox

\documentclass{article}
\usepackage{jf-all}
\pagesize{7.0in}{9in}

\newcommand{\ty}[1] {\jfverbtt{#1}}
\newcommand{\matlab} {\textsc{Matlab}\xspace}
\newcommand{\irt} {IRT\xspace}
\newcommand{\fatrix} {\texttt{Fatrix}\xspace}

\begin{document}

\title{
Image Reconstruction Toolbox
for \matlab
(and Freemat)
}
\author{Jeffrey A. Fessler\\
The University of Michigan
\\
\ty{fessler@umich.edu}
\date{\today}
}

\maketitle

\section{Introduction}

This is an initial attempt at documentation
of the image reconstruction toolbox (\irt)
for \matlab,
and any other \matlab emulator
that is sufficiently complete.
This documentation is, and will always be,
hopelessly incomplete.
The number of options and features
in this toolbox is near infinite.


\section{Overview}

\input s,over

\section{Special structures}

The \irt uses
two special custom-made object classes
extensively:
the \ty{strum} class,
which provides structures with methods,
and
the \fatrix class,
which provides a ``fake matrix'' object.
These objects exploit
\matlab's object oriented features,
specifically operator overloading.
The following overview of these objects
should help in
understanding the reconstruction code.

\input s,strum

\input s,fatrix

\end{document}
